

\begin{flashcard}[Definition]{Subset}
  Notation: $A \subseteq B$, informally $A \subset B$
  \vfill
  if $x \in A$, then $x \in B$
\end{flashcard}

\begin{flashcard}[Definition]{Proper Subet}
  Notation: $A \subset B$
  \vfill
  $A \subseteq B$ and $A \neq B$
\end{flashcard}

\begin{flashcard}[Definition]{Union}
  Notation: $A \cup B$
  \vfill
  $\{x \mid x \in A \text{ or } x \in B\}$
\end{flashcard}

\begin{flashcard}[Definition]{Intersection}
  Notation: $A \cap B$
  \vfill
  $\{x \mid x \in A \text{ and } x \in B\}$
\end{flashcard}

\begin{flashcard}[Definition]{Disjoint Sets}
  A and B are disjoint if $A \cap B = \O$
\end{flashcard}

\begin{flashcard}[Definition]{Equal Sets}
  A and B are equivalent if $A \subset B$ and $B \subset A$
\end{flashcard}

\begin{flashcard}[Definition]{Complement}
  For any set $A \subset U$
  \vfill
  $A' = \{x \in U \mid x \notin A \}$
\end{flashcard}

\begin{flashcard}[Definition]{Difference}
  $A \setminus B = A \cap B'$
\end{flashcard}

\begin{flashcard}[Definition]{Cartesian Product}
  $A \bigtimes B = \{ (a,b) \mid a \in A \text{ and } b \in B \}$
\end{flashcard}

\begin{flashcard}[Definition]{Relation}
  A relation from A to B is a subset of $A \bigtimes B$
\end{flashcard}

\begin{flashcard}[Definition]{Function or Map}
  Notation: $f: A \to B$, where A is the \emph{domain} and B is the \emph{target}
  \vfill
  A relation from A to B where $\forall a \in A, \exists!\ (a,b)$
  \vfill
  $f(a) = b$ means $(a,b) \in f$
\end{flashcard}

\begin{flashcard}[Definition]{Image of Function}
  $f(A) = \{f(a): a \in A\}$
  \vfill
  Note: $f(A) \subset B$
\end{flashcard}

\begin{flashcard}[Definition]{Surjective}
  Also known as \emph{onto}
  \vfill
  A function $f: A \to B$ for which $f(A) = B$
\end{flashcard}

\begin{flashcard}[Definition]{Injective}
  Also known as \emph{one-to-one}
  \vfill
  A function $f: A \to B$ for which $f(a_1) = f(a_2)$ implies $a_1 = a_2$
\end{flashcard}

\begin{flashcard}[Definition]{Bijective}
  A function which is both injective and surjective.
\end{flashcard}

\begin{flashcard}[Definition]{Composition}
  For $f: A \to B$ and $g: B \to C$ , the composition~$g \circ f$ is defined as:
  \vfill
  $(g \circ f)(x) = g(f(x))$
  \vfill
  Notes: $g \circ f: A \to C$, and composition is associative
\end{flashcard}

\begin{flashcard}[Definition]{Composition Properties}
  \begin{itemize}
    \item If g and f are surjective, $g \circ f$ is surjective
    \item If g and f are injective, $g \circ f$ is injective
    \item If g and f are bijective, $g \circ f$ is bijective
  \end{itemize}
\end{flashcard}

\begin{flashcard}[Definition]{Identity Map}
  Notation: $id_s$
  \vfill
  $id(x) = x, id: S \to S$
  \vfill
  Note: $id_s = \{ (x,x): x \in S \}$
\end{flashcard}

\begin{flashcard}[Definition]{Inverse}
  Notation: $\inv{f}$
  \vfill
  $g = \inv{f} \iff g \circ f = id_A$
  \vfill
  $f$ is invertible means it has an inverse.
  \vfill
  $f$ is invertible iff it's bijective.
\end{flashcard}

\begin{flashcard}[Definition]{Partition}
  Let X be a set. P is a partition of X means:
  \vfill
  $P = \{ p_i \subset X \mid P \neq \O \}$, all $p_i$ are disjoint, and the union of all $p_i$ is P.
\end{flashcard}

\begin{flashcard}[Definition]{Equivalence Relation}
  Relation R of X with these properties:
  \begin{itemize}
    \item Reflexive:  $\forall x \in X, (x,x) \in R$
    \item Symmetric:  $(x,y) \in R \implies (y,x) \in R$
    \item Transitive: $(x,y), (y,z) \in R \implies (x,z) \in R$
  \end{itemize}
  $x \sim y$ means $(x,y) \in R$
\end{flashcard}

\begin{flashcard}[Definition]{Equivalence Class}
  $[x] = \{y \in X \mid y \sim x\}$
  \vfill
  Two equivalence classes are either disjoint or equal.
\end{flashcard}

\begin{flashcard}[Definition]{Divisibility}
  Notation: $b \mid a$ (b divides a)
  \vfill
  $a = bk$ for some $k \in \Z$
\end{flashcard}

\begin{flashcard}[Definition]{Congruence}
  Let $r,s \in \Z$ and $n \in \N$.
  \vfill
  Notation: $r \equiv_n s$
  \vfill
  $n \mid (r - s)$
  \vfill
  Note: $\equiv_n$ is an equivalence relation of $\Z$
\end{flashcard}

\begin{flashcard}[Definition]{Properties of Congruence}
  \begin{itemize}
    \item $a + b \equiv b + a$ \quad $ab \equiv ba$
    \item $(a + b) + c \equiv a + (b + c)$ \quad  $(ab)c \equiv a(bc)$
    \item $a + 0 \equiv a$ \quad $a1 \equiv a$
    \item Multiplication distributes over addition
    \item $\exists b \in \Z \mid a + b \equiv 0 $
    \item Let $a \in \Z, a \neq 0$
      \newline
      $gcd(a,n) = 1 \iff \exists b \in \Z \mid ab \equiv_n 1$
  \end{itemize}
\end{flashcard}

\begin{flashcard}[Definition]{Mathematical Induction}
  Let $S(n)$ be a statement about $n \in \N$
  \vfill
  If $S(n_0)$ is true for some $n_0 \in \N$, and $S(k) \implies S(k+1)$, then $S(n)$ is true for all $n \geq n_0$
\end{flashcard}

\begin{flashcard}[Definition]{Strong Mathematical Induction}
  Let $S(n)$ be a statement about $n \in \N$
  \vfill
  If $S(n_0)$ is true for some $n_0 \in \N$, and $S(n_0), S(n_0 + 1), S(n_0+2) \dots S(k) \implies S(k+1)$, then $S(n)$ is true for all $n \geq n_0$
\end{flashcard}

\begin{flashcard}[Definition]{Well-Ordered}
  A non-empty subset of $\Z$ which contains a least element.
\end{flashcard}

\begin{flashcard}[Definition]{Well-Ordering Principle}
  Every non-empty subset of $\N$ is well-ordered.
\end{flashcard}

\begin{flashcard}[Definition]{Division Algorithm}
  Let $a,b \in \Z$ and $b > 0$
  \vfill
  $\exists!\ q,r \in \Z \mid a = qb + r$ and $0 \leq r < b$
\end{flashcard}

\begin{flashcard}[Definition]{Common Divisor}
  Let $a,b,d \in \Z$
  \vfill
  d is a common divisor of a and b means that $d \mid a$ and $d \mid b$
\end{flashcard}

\begin{flashcard}[Definition]{Greatest Common Divisor}
  Notation: $gcd(a,b)$ where $a,b \in \Z$
  \vfill
  Note: If $a,b > 0,\ \exists r,s \in \Z \mid gcd(a,b) = ar + bs$
\end{flashcard}

\begin{flashcard}[Definition]{Relatively Prime}
  $gcd(a,b) = 1$
\end{flashcard}

\begin{flashcard}[Definition]{Prime}
  p is prime if only numbers that divide it are 1 and p
  \vfill
  Otherwise, p is \emph{composite}
\end{flashcard}

\begin{flashcard}[Definition]{Prime Number Properties}
  $p \mid ab \implies p \mid a \text{ or } p \mid b$
  \vfill
  There exists an infinite number of primes.
\end{flashcard}

\begin{flashcard}[Definition]{Fundamental Theorem of Arithmetic}
  Let $n \in \N, n > 1$
  \vfill
  $n = p_1p_2p_3 \dots p_k$ where $p_i$ are prime
  \vfill
  This factorization is unique.
\end{flashcard}

\begin{flashcard}[Definition]{Binary Operation}
  A function $f: S \bigtimes S \to S$ on a set S
  \vfill
  $f(a,b)$ is denoted by $a \circ b$ or  $ab$
\end{flashcard}

\begin{flashcard}[Definition]{Group}
  Notation: $(S, \circ)$ for operation $\circ$ on set S
  \vfill
  Properties:
  \begin{itemize}
    \item Associative: $(ab)c = a(bc)$
    \item Identity Exists: $\exists e \in S \mid ai = a = ia, \forall a \in S$
    \item Inverses Exist: $\forall a \in S, \exists b \in S \mid ab = e = be$
  \end{itemize}
\end{flashcard}

\begin{flashcard}[Definition]{Group Properties}
  The inverses are unique for each element.
  \vfill
  $\inv{(ab)} = \inv{a}\inv{b}$
\end{flashcard}

\begin{flashcard}[Definition]{Commutative/Abelian Group}
  A group for which $ab = ba\ \forall a,b \in S$
\end{flashcard}

\begin{flashcard}[Definition]{Finite Group}
  $(G,\circ)$ is a finite group if G is a finite set.
  \vfill
  Otherwise, it's an infinite group.
\end{flashcard}

\begin{flashcard}[Definition]{Group Order}
  The order of finite group $(G,\circ)$ is $\mid G \mid$.
  \vfill
  The order of an infinite group is $\infty$.
\end{flashcard}

\begin{flashcard}[Definition]{Quaternion Group}
  $(Q_8, \circ)$ where $Q_8 = \{\pm 1, \pm i, \pm j, \pm k\}$ and:
  \begin{itemize}
    \item $i^2 = j^2 = k^2 = -1$
    \item 1 is identity
    \item -1 commutes everything
    \item $ij = k, jk = i, ki = j$
    \item $ik = -j, kj = -i, ji = -k$
  \end{itemize}
\end{flashcard}

\begin{flashcard}[Definition]{Matrix Group}
  $\M_2(\R)$ is multiplicative group for all 2x2 real number matrices.
  \vfill
  Non-abelian group.
\end{flashcard}

\begin{flashcard}[Definition]{General Linear Group}
  $GL_2(\R)$ is subset of $\M_2(\R)$ where every matrix is invertible.
  \vfill
  Non-abelian group.
\end{flashcard}

\begin{flashcard}[Definition]{Multiplicative Group Exponents}
  \begin{itemize}
    \item $g^n = g \circ g \circ g \circ g \dots \mid n \geq 1$
    \item $g^0 = e$
    \item $g^{-1} = \inv{g} \circ \inv{g} \circ \inv{g} \dots \mid n < 0$
    \item $g^mg^n = g^{m+n}$, $(g^m)^n = g^{mn}$
    \item $(gh)^n = (\inv{h}\inv{g})^{-n}$
  \end{itemize}
  \vfill
  If group is abelian, then $(gh)^n = g^nh^n$
\end{flashcard}

\begin{flashcard}[Definition]{Additive Group Exponents}
  \begin{itemize}
    \item $ng = g + g + g \dots \mid n \geq 0$
    \item $-ng = -g + -g + -g \dots \mid n < 0$
  \end{itemize}
  \vfill
  If group is abelian, then $m(g + h) = mg + mh$
\end{flashcard}

\begin{flashcard}[Definition]{Group Cancellation Laws}
  If G is a group, and $a,b,c \in G$, then:
  \vfill
  $ba = bc \implies a = c$ (left cancellation)
  \vfill
  $ab = cb \implies a = c$ (right cancellation)
\end{flashcard}

\begin{flashcard}[Definition]{Subgroup}
  A group $(H, \circ)$ is a subgroup of $(G, \circ)$ if $H \subseteq G$ and if H also forms a group under the operation~$\circ$. This is true iff:
  \vfill
  \begin{itemize}
   \item Identity $e \in G$ is in H
   \item $h_1,h_2 \in H \implies h_1 \circ h_2 \in H$
   \item $h_1 \in H \implies \inv{h_1} \in H$
  \end{itemize}
\end{flashcard}

\begin{flashcard}[Definition]{Subgroup Alternative Criteria}
  Let $H \subseteq G$. H is a subgroup of G iff:
  \vfill
  \begin{itemize}
   \item $H \neq \O$
   \item $a,b \in H \implies a \circ \inv{b} \in H$
  \end{itemize}
\end{flashcard}

\begin{flashcard}[Definition]{Trivial Subgroup}
  $\{e\}$
\end{flashcard}

\begin{flashcard}[Definition]{Proper Subgroup}
  H is a proper subgroup of G if H is a subgroup, and $H \subset G$
\end{flashcard}

\begin{flashcard}[Definition]{Proper Subgroup}
  H is a proper subgroup of G if H is a subgroup, and $H \subset G$
\end{flashcard}
